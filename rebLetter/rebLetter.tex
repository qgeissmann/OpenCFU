\documentclass{letter}
\usepackage{graphicx}


\newcommand{\Otitle}{\emph{OpenCFU, a New Free and Open-Source Software to Count Cell Colonies and
Other Circular Objects}}

\interfootnotelinepenalty=10000
\longindentation=0pt

\signature{	Quentin Geissmann
			PhD. student\\
Institute for Biology\\
Free University of Berlin\\
}

\address{
Institute for Biology\\
Free University of Berlin\\
K\"{o}nigin-Luise-Strasse 1-3\\
D-14195 Berlin, Germany\\
q.geissmann@fu-berlin.de}
\name{Quentin Geissmann}


\begin{document}
 
\begin{letter}{}
\opening{Dear editors,}
 Thank you for having considered my manuscript entitled ``\Otitle{}'' for publication in PLoS ONE.
 In response to your suggestion of ``Major Revision'', and thanks to the very useful criticisms of the two anonymous reviewers, 
 I have undertaken extensive modifications both in my manuscript and in OpenCFU; the software it details.
 In this document, I will attempt to answer to each criticisms addressed to my original manuscript and explain the revisions they have elicited.
%~~~~~~~~~~~~~~~~~~~~~~~~~~~~~~~~~~~~~~~~~~~~~~~~~~~~~~~~~~~~~~~~~~~~~~~~~~~~~~~~~~~~~~~~~~~~~~~~~~%

 Fist of all, it was requested that I compare my method against additional alternatives:
  \begin{quote}
 Academic editor:\\
 ``[\ldots] but what about other tools, e.g. those included in ImageJ, those based on MMorph. 
 Please analyze the performance increase of your software over these and the other available alternatives.''\\
 Reviewer 2:\\
 ``In the introduction refs to comparable software is missing.''
  \end{quote} 
 I have to admit that this was valid and justified criticism.
 My revised manuscript now includes a comparison to the ImageJ macro recently proposed by Cai \emph{et al.}.
 As I explain in my modified introduction I have, despite my efforts, faced many difficulties trying to satisfy this demand any further.
 Here, I describe other alternatives I have considered without being able to compare them to OpenCFU:

\begin{itemize}
  
	\item Sofware works that were part of an integrated colony counter for which the authors did not provide an independant implementation\footnote{For example: Zhang, Chengcui, and Wei-Bang Chen. ``An Effective and Robust Method for Automatic Bacterial Colony Enumeration.'' In International Conference on Semantic Computing, 2007. ICSC 2007, 581–588, 2007.}.
	
	\item Theoretical publications that did not refer to any implementation\footnote{For example: Kaur, Gurpreet, and Poonam Sethi. ``A Novel Methodology for Automatic Bacterial Colony Counter.'' International Journal of Computer Applications 49, no. 15 (July 28, 2012): 21–26.}. 
	
   \item Tools that were not available. Namely, ``ScanCount''\footnote{Dahle,
   Jostein, Manish Kakar, Harald B. Steen, and Olav Kaalhus. ``Automated
   Counting of Mammalian Cell Colonies by Means of a Flat Bed Scanner and Image
   Processing." Cytometry Part A 60A, no. 2 (2004): 182–188.}
  (based on ``Analysis Pro 3.0''),
 ``CHiTA''\footnote{Bewes, J M, N Suchowerska, and D R McKenzie. ``Automated
 Cell Colony Counting and Analysis Using the Circular Hough Image Transform
 Algorithm (CHiTA)." Physics in Medicine and Biology 53, no. 21 (November 7,
 2008): 5991–6008.} (based on MATLAB) or ``Arraycount''\footnote{Kachouie,
 Nezamoddin N., Lifeng Kang, and Ali Khademhosseini. ``Arraycount, an Algorithm
 for Automatic Cell Counting in Microwell Arrays.'' BioTechniques 47, no. 3
 (September 2009): x–xvi.} (based on MATLAB) could not be obtained. No reference
 to download links were specified, and tools seemed absent from the supplementary
 materials.
 Further investigations involving trying to visit the authors respective
 websites, running web searches and  attempting to directly
 contact the corresponding authors were equally unsuccessful.
 
 \item ``Clono-counter''\footnote{Niyazi, Maximilian, Ismat Niyazi, and Claus Belka. ``Counting Colonies of Clonogenic Assays by Using Densitometric Software.'' Radiation Oncology 2, no. 1 (January 9, 2007): 4.}.
 Despite its availability, the program was affected by a very
 serious bug that seems to make the analysis of images larger than the screen
 dimensions impossible.
 The software had many other important flaws including the impossibility to open multiple
 files in the same program instance.
    
 \item The plug-in, partially included in ImageJ, developed by Sieuwerts \emph{et al.}\footnote{Sieuwerts, S., F. A.M De Bok, E. Mols, W. M De Vos, and J. E.T Van Hylckama Vlieg. ``A Simple and Fast Method for Determining Colony Forming Units.'' Letters in Applied Microbiology 47, no. 4 (October 1, 2008): 275–278.}.
 The method relies on general threshold which performed
 poorly because the background intensity is not constant (within and between
 images). In addition, it requires, according to the authors, manual conversion to grey-scale and thresholding of each image before counting particles.
 This steps seems to me very subjective, error-prone and time-consuming.
 
 \item Cell-Profiler pipeline proposed by Vokes and
 Carpenter\footnote{Vokes, Martha S, and Anne E Carpenter. ``Using CellProfiler
 for Automatic Identification and Measurement of Biological Objects in Images.''
 Current Protocols in Molecular Biology Edited by Frederick M Ausubel Et Al
 Chapter 14, no. April (2008): Unit 14.17.}. I could not obtain any satisfying
 results by performing reasonably minor changes (including inverting the image,
 changing the block-size of the illumination correction functions, or, for
 instance, the relative weight of the colour channels). Further alteration of the proposed method would be considered as
 developing a new method and my point is not to compare OpenCFU to other
 methods developed by myself.
 \end{itemize}
 % ~~~~~~~~~~~~~~~~~~~~~~~~~~~~~~~~~~~~~~~~~~~~~~~~~~~~~~~~~~~~~~~~~~~~~~~~~~~~~~~~~~~~~~~~~~~~~~~~~~%
 
	Then, criticisms for simply implementing existing method rather than providing a novel algorithm were raised:
  \begin{quote}
   Reviewer \#1:\\
 ``The image processing is very comparable to previously published results''
 \end{quote}
 I acknowledge this fact and think this criticisms were valid. Although it was very fast, the method
 described in the first manuscript, was rather simplistic and overall lacked robustness. 
 A very significant part of my revisions has involved the development of a new exhaustive
 algorithm that focuses on robustness and accuracy wilst preserving a reasonably fast processing speed.\\
 %~~~~~~~~~~~~~~~~~~~~~~~~~~~~~~~~~~~~~~~~~~~~~~~~~~~~~~~~~~~~~~~~~~~~~~~~~~~~~~~~~~~~~~~~~~~~~~~~~~%
 
 It was also asked that I make clear the advantages of my software over existing solutions:
 \begin{quote}
 Academic editor:\\
 ``Both reviewers [\ldots] criticize your paper for not making clear in what
  way the software is an advantage over existing tools''\\
  ``In terms of validation, it is not clear if the method is more accurate than the alternatives?''\\
 \end{quote}
In my first manuscript, I presented processing speed as the main advantage of OpenCFU over NICE.
However, one could legitimately argue that, on modern hardware, NICE is fast enough since the processing time is negligible compared to, for instance, the total acquisition time.
For this reason, I have also taken out ``table~1'', which compared the time to process all the plates for different agents/methods.
I do however still maintain that speed is important, especially if the method needs a lot of calibration and rectifications.
My revised manuscript, now describes and quantifies other advantages of OpenCFU over alternatives. That is its accuracy and robustness.\\
 %~~~~~~~~~~~~~~~~~~~~~~~~~~~~~~~~~~~~~~~~~~~~~~~~~~~~~~~~~~~~~~~~~~~~~~~~~~~~~~~~~~~~~~~~~~~~~~~~~~%
 
They were discrepencies in the reviewers comments regarding the advantages of the user interface:
  \begin{quote}
  Reviewer \#1:\\
	  ``Furthermore, it [OpenCFU] comes with a clean graphical user interface making it user-friendly to biologists.
	   Another nice addition is that the interface allows setting a number of parameter values which directly affect the image segmentation.''\\
   Reviewer \#2:\\
		``There is no further reference to GUI tools or GUI evaluation that makes it, at least, plausible that, 
		in this case, this contributes to the success of OpenCFU.''
	\end{quote}
I my first manuscript, I did not ellaborate on the features and advantages of the GUI as 
I thought this would be assessed by each OpenCFU user. 
In my revised submission, I now include a section that describes some of the features of the user interface,
 and provide explanations as to why they provide an advantage over alternatives.\\
 
 %~~~~~~~~~~~~~~~~~~~~~~~~~~~~~~~~~~~~~~~~~~~~~~~~~~~~~~~~~~~~~~~~~~~~~~~~~~~~~~~~~~~~~~~~~~~~~~~~~~%
 It was asked that I describe the method used with more precision:
 \begin{quote}
	 Reviewer \#2:\\
	 ``a watershed approach for touching subjects was used. 
	what criteria were applied for the "untouching" of the objects for the count?
	there is no ref. to the watershed method for objects based on the DT;''\\
	``i miss a flow-chart of the method.''\\
	``what distance transform was used ; this influences the speed considerably.''
 \end{quote}
 My revised submission incorporates a flowchart (in a new figure) of the novel method.
 Moreover, the section ``Materials and Methods'' now contains a detailed description of the procedure implemented to split and accept objects as colonies.\\
%~~~~~~~~~~~~~~~~~~~~~~~~~~~~~~~~~~~~~~~~~~~~~~~~~~~~~~~~~~~~~~~~~~~~~~~~~~~~~~~~~~~~~~~~~~~~~~~~~~%

At the time of the first submission, OpenCFU could produce only a summary output which elicited some criticisms:
  \begin{quote}
  Academic editor:\\
   ``The software produces quite limited output; other outputs could include colony size, shape, density.''\\
   Reviewer \#1:\\
   ``Also the output is relatively limited. More parameters could easily be extracted from the individual colonies.''\\
	\end{quote}
 In response to this, the software now includes a ``detailed output'' option which gives information about each colony.\\
%~~~~~~~~~~~~~~~~~~~~~~~~~~~~~~~~~~~~~~~~~~~~~~~~~~~~~~~~~~~~~~~~~~~~~~~~~~~~~~~~~~~~~~~~~~~~~~~~~~%

	Some more details about the use of capture devices and its justification were requested:
	 \begin{quote}
	 Academic editor:\\
	``I also miss a detailed analysis of the use with webcam images. Is it fast enough to use it for counting moving objects on webcams?''\\
	 Reviewer \#2:\\
	``inclusion of the usage of realtime counting with a webcam seems not useful to me.
	 One can argue to use a webcam in these applications but why?
	It is better to use a multi-purpose imaging system with a high NA lens. 
	The is what webcams are not made for. 
	And moreover, the images include more noise.''\\	
 \end{quote}
 The new algorithm being slower, it does not allow ``real-time'' processing any more. 
In addition, users had reported that real-time counting was confusing and somehow unnecessary.
 In the updated version of the program, the video stream is still displayed, but the user has to click on a button to stop it which; will automatically analyse the last frame.
 I think it is interesting to have the option of a low-cost integrated acquisition system in the software.
This has been an important matter illustrated by the exhaustive focus of authors on using desktop scanners.
 Biological material such as bacteria is often confined to specific rooms or laboratories which makes is hard to share equipment between groups and could encourage low-cost solutions.
 In this study I show that using a simple webcam can lead to very good results, but OpenCFU is, \emph{a priori}, compatible with other capture devices such
 as UVC compliant ones (for instance some USB microscopes) and firewire cameras. \\
%~~~~~~~~~~~~~~~~~~~~~~~~~~~~~~~~~~~~~~~~~~~~~~~~~~~~~~~~~~~~~~~~~~~~~~~~~~~~~~~~~~~~~~~~~~~~~~~~~~%

It was recommended that I provide informations about the variation between human counts:
	 \begin{quote}
	 Reviewer \#2:\\
	 ``for the human exp. the inter/intro subject variation is interesting to know. this shows how expert the `group' was.''
	 \end{quote}
In my revised manuscript, I use the deviations from the reference as a measure of error.
This contrasts with my first manuscript in which I only used the $R^2$ of the linear regression to the reference. 
The error variability between humans is now taken in consideration.  
Figure 4B in the revised manuscript also represents and compares the respective errors of the ``best'' and ``worst'' 
humans to the human average error.\\
%~~~~~~~~~~~~~~~~~~~~~~~~~~~~~~~~~~~~~~~~~~~~~~~~~~~~~~~~~~~~~~~~~~~~~~~~~~~~~~~~~~~~~~~~~~~~~~~~~~%

 Questions about the use of colours were raised:
   \begin{quote}
	 Reviewer \#2:\\
	 ``can using a grey value camera speed up the process?''\\
	 ``why refer to hue of 80; this is another intensity representation
and one converts to greyvalues.''
``at the onset the color information is thrown away.''\\
``so why color anyway?''\\
 \end{quote}
 In its current state, OpenCFU will not be faster with a grey-scale image. 
 Considering that the most time consuming part of the new algorithm occurs after generating the grey-scale image, 
 I predict only a marginal time-gain by implementing a specific pre-processing step for grey-scale images.
 Even if the image is converted to grey-scale early, it is kept in memory. 
 It is therefore possible to remap detected colonies onto the original image in order to obtain per-colony colour informations.
 As I explain in the revised discussion, using the colour of detected objects for post-processing can have, at least, two advantages: 
 it increases robustness by allowing the exclusion of some foreign objects such as writing or contaminant colonies, 
 it could also be used to segregate populations of cells which can be useful in experiments such as ``blue white screening''.\\
 %~~~~~~~~~~~~~~~~~~~~~~~~~~~~~~~~~~~~~~~~~~~~~~~~~~~~~~~~~~~~~~~~~~~~~~~~~~~~~~~~~~~~~~~~~~~~~~~~~~%
 
I have been requested to provide more informations about the operating system I
used for testing:
    \begin{quote}
	 Reviewer \#2:\\
	 ``mention the Linux brand. eg. Ubuntu/Fedora + vs.''
 \end{quote}
My revised manuscript now mention these informations. In addition I refer to the version of OpenCV and the \texttt{C++}
 compiler I used since  they are more likely to impact on speed than the OS distribution/version.\\
 %~~~~~~~~~~~~~~~~~~~~~~~~~~~~~~~~~~~~~~~~~~~~~~~~~~~~~~~~~~~~~~~~~~~~~~~~~~~~~~~~~~~~~~~~~~~~~~~~~~%
 
A better explanation about the former\footnote{The figure 2 of the original submission correspond to the figure 3 of the revised one.} 
  figure 2 in the text was requested: 
    \begin{quote}
	 Reviewer \#2:\\
	 ``stiched or scaled up? this is not well explained in the text?''
 	\end{quote}
  I provide more information in my present manuscript.
  I also rephrased the description and substituted the word ``stitched'' by  ``tiled'' which I think is more appropriate.\\
%~~~~~~~~~~~~~~~~~~~~~~~~~~~~~~~~~~~~~~~~~~~~~~~~~~~~~~~~~~~~~~~~~~~~~~~~~~~~~~~~~~~~~~~~~~~~~~~~~~% 

Issues about the former figure 2b were raised:
 \begin{quote}
	 Reviewer \#2:\\
	``this is a histogram and not a continues line.
the values of image size are discrete. a trendline can be used but it should 
be indicated as such.''
 \end{quote} 
I am afraid not to understand how I could represent my bivariate data by an histogram.
The figure legend of the revised manuscript now states that the points are the data and the the line segments are only present to aid understanding.
 I believe this representation to be fairly common\footnote{
 For instance, figure 10 in He, Kaiming, Jian Sun, and Xiaoou Tang. ``Fast Matting Using Large Kernel Matting Laplacian Matrices.'' In 2010 IEEE Conference on Computer Vision and Pattern Recognition (CVPR), 2165 –2172, 2010.\\
 And figures 12-17 in Chang, Fu, Chun-Jen Chen, and Chi-Jen Lu. ``A Linear-time Component-labeling Algorithm Using Contour Tracing Technique.'' Computer Vision and Image Understanding 93, no. 2 (February 2004): 206–220.
 }.\\
  %~~~~~~~~~~~~~~~~~~~~~~~~~~~~~~~~~~~~~~~~~~~~~~~~~~~~~~~~~~~~~~~~~~~~~~~~~~~~~~~~~~~~~~~~~~~~~~~~~~% 
 
	A question regarding the alternative of improving NICE was asked:
 \begin{quote}
	 Reviewer \#2:\\
	``can tuning of NICE lead to higher speed?''
 \end{quote} 
I am no expert in MATLAB and I do not know to what extent this is possible.
I developed a new software rather that trying to improve NICE because mainly NICE is based on MATLAB.
In order to be effectively distributed, it should be compiled as a standalone executable (relying on MCR).
This comes with many limitations such as slower execution time, high memory footprint and strict dependency to a specific version of the MCR.
In addition, as pointed out by PLoS ONE editorial policies\footnote{http://www.plosone.org/static/editorial.action\#methods subsection Availability},
 complete open-source solution should be preferred (and MATLAB is not open-source).\\
%~~~~~~~~~~~~~~~~~~~~~~~~~~~~~~~~~~~~~~~~~~~~~~~~~~~~~~~~~~~~~~~~~~~~~~~~~~~~~~~~~~~~~~~~~~~~~~~~~~% 
 
 Finally, I was advised to provide biological data in addition to my software:
 \begin{quote}
	 Reviewer \#1:\\
	``Furthermore, it would be nice if the presentation of the software was backed-up with some biological results.''
 \end{quote} 
 This is beyond the scope of this study.\\ 
%~~~~~~~~~~~~~~~~~~~~~~~~~~~~~~~~~~~~~~~~~~~~~~~~~~~~~~~~~~~~~~~~~~~~~~~~~~~~~~~~~~~~~~~~~~~~~~~~~~%

I would like to bring few precisions about the supplementary material I provide on the web-page of OpenCFU\footnote{http://sourceforge.net/projects/opencfu/files/}
and how I intend to maintain the program:\\
The version of the program (3.3) used in the publication will be considered as a ``milestone'' and will therefore never be deleted from the website (even if many updates follow).\\
Since I lacked time and did not think it was useful for the reviewers, I did not update the user-manual or the video tutorial from the previous version.
However, if my submission is accepted, I will engage myself to maintain both of them.\\
A list of images is provided in order to help calibration and to attest the versatility of the program. 
I hope to extend the diversity of this list in the near future.\\
New features will be included in future versions of OpenCFU.
For instance, support for multiple ROIs, bash mode, a version without GUI (for use in clusters) and supervised outliers exclusion.
User feedback and criticism will hopefully help to drive these updates.\\

In conclusion, I have taken into serious consideration all the criticisms
and questions raised by the academic editor and the two anonymous reviewers.
I have included comparisons with other tools beside NICE and I have developed and
implemented a novel algorithm. I have also provided new results, figures and analysis
that demonstrate the advantages provided OpenCFU.  Finally, I have included in my revised
manuscript additional and complementary information requested by the
reviewers.
Altogether, the suggested revisions have, I believe, improved the quality of my
work and made my manuscript more compliant to PLoS ONE acceptance criteria regarding software related publications.



\closing{Yours Faithfully,
     %\includegraphics[width=1.5in]{sheffield.eps
     }
\ps{\emph{p.s.} Almost all parts of my manuscript, except the first part of the introduction, have been rewritten. 
As a consequence, the text bears little similarity to the original version.
 For this reason, I judged it was not possible or appropriate to provide a marked-up version.}

\end{letter}

 %\includegraphics[width=2in]{}
\end{document}
















